\section{Кубит на нейтральных атомах $^{87}\text{Rb}$}
\label{sec:chapter_2}

В этой главе обсуждается устройство кубита на нейтральном атоме $^{87}\text{Rb}$: загрузка атомов в массив оптических пинцетов, инициализация состояния кубита, однокубитные и двухкубитные логические операции, а также считывание состояния.


% \subsection{Охлаждающие переходы}
\subsection{Массив оптических пинцетов}
% Про сортировку и перемещение можно полностью в массиве оптических пинцетов написать

Одним из преимуществ платформы на нейтральных атомах является то, что из атомов, в которые кодируется кубит, можно формировать произвольные одномерные и двумерные структуры, а также легко переключаться между ними. Делается это с помощью пространственного модулятора света (SLM) и акустооптического дефлектора (АОД) в скрещенной конфигурации. SLM представляет собой прямоугольную матрицу из жидких кристаллов, на каждый кристалл которой можно независимо подавать напряжение и, тем самым, менять показатель преломления за счёт эффекта двулучепреломления. Таким образом, с помощью SLM можно формировать произвольную фазовую маску(голограмму), ограниченную лишь размерами матрицы и размером пикселя. С помощью голограммы можно преобразовать падающий на неё лазерный луч в двумерный массив оптических пинцетов произвольной конфигурации, в который далее можно загрузить атомы. Сами фазовые маски можно рассчитать с помощью алгоритма Герчберга-Сакстона \cite{Gerchberg1972APA} и его модификаций \cite{robust_masks,Zupancic:16}. Для загрузки атомы из магнитооптической ловушки (МОЛ) перегружаются в двумерный массив оптических пинцетов \cite{Ashkin:99}, сформированный жёстко-сфокусированными гауссовыми пучками с радиусом перетяжки порядка $1 \text{ мкм}$. Если использовать оптические пинцеты в режиме столкновительной блокады \cite{Schlosser_Grangier,Kuppens_Wieman}, при которой сильно возрастают двухчастичные потери, то при загрузке атомов из МОЛ в дипольной ловушке примерно с одинаковой вероятностью оказывается либо один атом, либо ноль. Этот механизм позволяет ловить одиночные атомы, а затем кодировать в них кубит. 

\begin{figure}[ht]
	\centering
	\includegraphics[width=0.5\textwidth]{images/Array49.png}
	\caption{Фотография полностью заполненного массива из 49 атомов. По краям массива находятся дополнительные резервные ловушки. }
	\label{fig:array49}
\end{figure}

\subsection{Инициализация состояния}

Чтобы что-то делать с кубитами, нужно уметь инициализировать их начальное состояние. В нашей системе в качестве кубитных состояний используются магнитные подуровни сверхтонкого расщепления основного состояния $^{87}\rm{Rb}$, $\ket{0}=\ket{5^2S_{1/2},F=1,m_F=0}, \; \ket{1}=\ket{5^2S_{1/2},F=2,m_F=0}$, поэтому нам нужно уметь помещать атом в одно из этих состояний. Для инициализации используется схема, показанная на рисунке \ref{fig:initialization}.

\begin{figure}[ht]
	\centering
	\includegraphics[width=0.9\textwidth]{images/initialization_scheme.pdf}
	\caption{Населённость с уровня $\ket{1}$ не уходит на другие уровни, так как переход $\ket{5^2\rm{P}_{1/2},F=2,m_F=0}\leftrightarrow \ket{5^2\rm{P}_{1/2},F=2,m_F=0}$ запрещён правилами отбора для дипольных переходов, поэтому вся населенность в итоге сваливается на уровень $\ket{1}$.}
	\label{fig:initialization}
\end{figure}

С помощью двух лазеров с линейной и правой циркулярной поляризацией атомы из терма $5^2\rm{S}_{1/2}$ с $m_F\neq0$ перекачиваются на терм $5^2\rm{P}_{1/2}$, а затем спонтанно распадаются обратно на $5^2\rm{S}_{1/2}$, причем часть населенности попадает на уровень $\ket{1}=\ket{5^2\rm{S}_{1/2},F=2,m_F=0}$. Так как переход $\ket{5^2\rm{P}_{1/2},F=2,m_F=0}\leftrightarrow \ket{5^2\rm{P}_{1/2},F=2,m_F=0}$ дипольно-запрещён правилами отбора по чётности \cite{Belousov}, то населённость с него не уходит на терм $5^2\rm{P}_{1/2}$, а значит после нескольких итераций такой последовательности вся населенность останется на уровне $\ket{1}$. Естественная ширина и время жизни терма $5^2\rm{P}_{1/2}$ составляют примерно $\Gamma = 2\pi \times 6 \text{ МГц}$ и $170 \text{ нс}$ \cite{Rb87}, для накачки используется мощный пучок с $\Omega \gg \Gamma$, поэтому скорость инициализации начального состояния определяется временем жизни $5^2\rm{P}_{1/2}$. Чтобы минимизировать ошибку приготовления состояния, последовательность выполняется в течение $5 \text{ мс}$. Также существует схема инициализации начального состояния с помощью рамановских пучков \cite{toffoli}, в дальнейшем планируется её использование. 


\subsection{Однокубитные логические операции}

Однокубитные логические операции производятся с помощью возбуждения осцилляций Раби в двухуровневой системе, образованной двумя выделенными энергетическими уровнями атома $^{87}\rm{Rb}$. Пусть уровни $\ket{0}$ и $\ket{1}$ отстоят по энергии на $\hbar \omega_0$, на атом светит переменное классическое поле с частотой $\omega$, тогда гамильтониан системы атом + поле в приближении вращающейся волны можно записать в виде(гамильтониан Раби, $\hbar=1$) \cite{Lukin, Steck}

\begin{equation}
	H = \frac{\Delta}{2} \sigma_z + \frac{\;\rm{Re}(\Omega)}{2}\sigma_x + \frac{\;\rm{Im}(\Omega)}{2}\sigma_y= \frac{1}{2}\vec{\Omega}\cdot\vec{\sigma},
\end{equation}

где $\Delta = \omega - \omega_0$ - отсройка от резонанса, $\Omega$ - частота Раби, которая выражается через матричные элементы соответствующего оператора из мультипольного разложения \cite{LL_teorpol,Steck}. $\sigma_i$ - матрицы Паули, которые в кубитном базисе выражаются как

\begin{equation}
	\sigma_x = \ket{0}\bra{1} + \ket{1}\bra{0}, \; \sigma_y = i\ket{1}\bra{0} - i\ket{0}\bra{1}, \; \sigma_z = \ket{0}\bra{0}-\ket{1}\bra{1}.
\end{equation}

Оператор эволюции записывается в виде 

\begin{equation}
	U(t) = \exp\left(-\frac{it}{2}\vec{\Omega}\cdot\vec{\sigma}\;\right) = \exp\left(-i\frac{\tilde{\Omega}t}{2}\vec{n}\cdot\vec{\sigma}\right),
	\label{eq:evolution_operator}
\end{equation}

то есть мы получили вращение на сфере Блоха на угол $\tilde{\Omega}t$ вокруг вектора $\vec{n}$, которые выражаются как

\begin{equation}
	\vec{n}=\left(\frac{\rm{Re(\Omega)}}{\sqrt{|\Omega|^2 + \Delta^2}},\frac{\rm{Im}(\Omega)}{\sqrt{|\Omega|^2 + \Delta^2}},\frac{\Delta}{\sqrt{|\Omega|^2 + \Delta^2}}\right),
\end{equation}

\begin{equation}
	\tilde{\Omega} = \sqrt{|\Omega|^2 + \Delta^2}.
\end{equation}

Отсюда сразу видно, что для экспериментальной реализации $X$ и $Y$ гейтов можно посветить на атом резонансным $\Delta = 0$ полем с фазой $0$ и $\pi/2$ соответственно в течение времени $t$ такого что $\Omega t = \pi$. Для реализации $Z$-гейта можно посветить на атом сильно-отстроенным пучком, 

Сейчас на нашей установке можно совершать однокубитные операции с помощью радиочастотного поля, возбуждающего магнитодипольный переход между кубитными состояними, либо с помощью оптических рамановских двухфотонных переходов. Рассмотрим для начала реализацию однокубитных гейтов с радиочастотным возбуждением.

\subsubsection{Однокубитные операции с СВЧ-возбуждением}

Одной из схем реализации однокубитных вентилей является использование резонансной СВЧ-антенны, которая возбуждает магнитодипольный переход между кубитными состояниями $\ket{0}=\ket{5^2S_{1/2},F=1,m_F=0}$, $\ket{1}=\ket{5^2S_{1/2},F=2,m_F=0}$ на частоте $6.8 \text{ ГГц}$. Так как СВЧ-излучение засвечивает сразу весь атомный массив (расстояние между атомами порядка $3$ мкм), то требуется дополнительный адресующий лазер для реализации локальных однокубитных операций(рис. \ref{fig:shf_scheme}). 

\begin{figure}[ht]
	\centering
	\includegraphics[width=0.8\textwidth]{images/shf_gates.pdf}
	\caption{Схема реализации адресных гейтов с помощью СВЧ-антенны и дополнительного адресующего лазера. Все атомы, кроме целевого, находятся вне резонанса с СВЧ-антенной. Целевой атом находится в резонансе за счёт оптических штарковских сдвигов от адресующего пучка.}
	\label{fig:shf_scheme}
\end{figure}

СВЧ-антенна отстраивается от кубитной частоты так, чтобы подавить осцилляции Раби на всех атомах. Для целевого же атома адресующий пучок сдвигает частоту перехода за счёт оптического штарковского сдвига(AC Stark shift), возвращаёт целевой атом в резонанс с антенной. Так реализуются локальные однокубитные операции с помощью СВЧ-антенны. К недостаткам локальных СВЧ-гейтов можно отнести их большую длительность за счёт делокализации излучения по всему массиву, нагрев атома за счёт мощного адресующего пучка. Таким образом, СВЧ-излучение больше подходит для реализации глобальных однокубитных операций сразу над всем атомным массивом, что тоже иногда требуется делать. 

На текущий момент точность глобальных СВЧ-гейтов составляет $F=(99.95 \pm 0.07)\%$. Оценка точности однокубитных операций производится с помощью протокола randomized benchmarking \cite{Hines_2023,Proctor:2022aa}. К факторам, ограничивающим точность СВЧ-гейтов можно отнести дифференциальные штарковские сдвиги в оптической ловушке, выход из резонанса за счёт эффекта Доплера при движении атома в оптической ловушке. Точность локальных однокубитных операций с СВЧ-возбуждением достаточно низкая, более перспективным подходом является использование двухфотонных рамановских переходов с полностью оптическим возбуждением. 
В данной работе моделирование точности СВЧ-гейтов не производится, потому что они уже работают достаточно хорошо, чего до начала работы нельзя было сказать про однокубитные вентили с рамановским двухфотонным возбуждением. 

\subsubsection{Однокубитные операции с оптическим возбуждением}

Для реализации однокубитных вентилей с рамановским двухфотонным возбуждением в $\Lambda$-конфигурации используется схема атомных уровней и импульсная последовательность, показанные на рисунках \ref{fig:atom_scheme_raman} и \ref{fig:pulse_scheme_raman}. Использование полностью оптического рамановского возбуждения позволяет фокусировать излучение на отдельные атомы и совершать локальные однокубитные операции. Подробное обсуждение рамановского возбуждения содержится в главе \ref{sec:chapter_3}.

\begin{figure}[H]
	\centering
	\includegraphics[width=0.9\textwidth]{images/atom_scheme_raman.pdf}
	\caption{Схема уровней $^{87}\text{Rb}$ и использующиеся переходы.}
	\label{fig:atom_scheme_raman}
\end{figure}

\begin{figure}[H]
	\centering
	\includegraphics[width=0.6\textwidth]{images/pulse_scheme_raman.pdf}
	\caption{Импульсная последовательность для реализации рамановского двухфотонного возбуждения.}
	\label{fig:pulse_scheme_raman}
\end{figure}

\subsection{Двухкубитные логические операции}

Для реализации запутываюших операций используется дополнительный высоковозбужденный уровень с главным квантовым числом порядка $50-100$, называемый ридберговским состояним $\ket{r}$. Для таких состояний возникает эффект ридберговской блокады \cite{PhysRevLett.85.2208}, при котором два нейтральных атома, возбужденных в ридберговское состояние на расстоянии порядка микрометра, начинают сильно взаимодействовать диполь-дипольным образом. Если $R$ - расстояние между атомами, а $n$ - главное квантовое число, то энергия диполь-дипольного взаимодействия между двумя водородоподобными нейтральными атомами во втором порядке невырожденной теории возмущений записывается как $V\sim \frac{1}{R^{6}}$. Способ реализации двухкубитных гейтов можно проиллюстрировать следующим образом. Допустим, что мы приготовили два атома в состояние $\ket{11}$ и начали одновременно светить на них лазером в резонансе с переходом $\ket{1}\leftrightarrow\ket{r}$ с частотой Раби $\Omega$. Состояние $\ket{rr}$ при этом смещено на величину ридберговского взаимодействия $V$, схема уровней показана на рисунке \ref{fig:rydberg_blockade}.

\begin{figure}
	\centering
	\includegraphics[width=0.75\textwidth]{images/rydberg_blockade_scheme.pdf}
	\caption{За счёт эффекта ридберговской блокады атомы не могут одновременно возбудиться в ридберговское состояние, из-за чего возникают осцилляции между невозбужденным состоянием $\ket{11}$ и суперпозицией $\frac{1}{\sqrt{2}}\left(\ket{1r} + \ket{r1}\right)$, в которой лишь один из атомов находится в ридберговском состоянии.}
	\label{fig:rydberg_blockade}
\end{figure}

Если расстояние между атомами достаточно маленькое, то при одновременном воздействии резонансным полем на переход $\ket{1}\leftrightarrow\ket{r}$ атомы не могут одновременно возбудиться в ридберговское состояние $\ket{rr}$, так как оно сдвинуто по энергии на $V \gg \Omega$. Из-за этого возникают осцилляции Раби между невозбужденным состоянием $\ket{11}$ и суперпозицией $\frac{1}{\sqrt{2}}\left(\ket{1r} + \ket{r1}\right)$, в которой возбужден лишь один из атомов. Характерное расстояние, на которое нужно сблизить атомы, чтобы наблюдался эффект ридберговской блокады, определяется как $R_B = \left(\frac{C_6}{\Omega}\right)^{1/6}$, называется радиусом ридберговской блокады и составляет примерно $10 \;\text{мкм}$ для $\ket{r}=\ket{70 S_{1/2}}$ и $\Omega = 2\pi \times 1 \;\text{МГц}$. Видно, что за счет такого механизма можно приготовить максимально запутанное состояние Бэлла в базисе $\ket{1}, \ket{r}$. Подбором правильной импульсной последовательности можно получить нативный $CZ$-гейт в кубитном базисе за счёт эффекта ридберговской блокады, аналогично можно делать многокубитные гейты \cite{toffoli}, размещая несколько атомов в радиусе ридберговской блокады. Более подробно двухкубитные гейты будут обсуждаться в главе \ref{sec:chapter_4}.

\subsection{Считывание состояния}

После проведения логических операций хочется уметь считывать получившееся состояние кубитов. Для этого используется мощный выбивающий пучок push-out (рис. \ref{fig:main_setup}), который производит нагрев атома, если тот находится в состоянии $\ket{1}$. Нагрев происходит за счёт перекачки атома в циклический переход $\ket{5^2S_{1/2},F=2,m_F=2}\leftrightarrow\ket{5^2P_{3/2},F=3,m_F=3}$, после нескольких осцилляций Раби на этом переходе атом вылетает. По вероятности выживания атома в оптическом пинцете определяется его состояние. Стоит отметить, что также существуют неразрушающие схемы измерений состояния атома \cite{QND} (QND measurements), однако они имеют меньшую точность. Недавно также было продемонстрирована неразрущающая схема измерений для щёлочноземельного атома $^{171}\text{Yb}$ \cite{Ma:2023aa} за счёт использования дополнительного метастабильного возбужденного состояния.

\subsection{Цикл работы квантового компьютера на нейтральных атомах}



\begin{figure}[ht]
	\centering
	\includegraphics[width=1.0\textwidth]{images/Main_setup_vs_Raman_EN.pdf}
	\caption{Принципиальная схема экспериментальной установки.}
	\label{fig:main_setup}
\end{figure}

Принципиальная схема экспериментальной установки квантового компьютера на холодных нейтральных атомах $^{87}\text{Rb}$ представлена на рисунке $\ref{fig:main_setup}$. Основные параметры нашей установки и длительность различных этапов работы указаны в таблицах \ref{tab:times} и \ref{tab:params}. Цикл работы квантового компьютера на холодных нейтральных атомах $^{87}\text{Rb}$ состоит из следующих основных этапов:

\begin{enumerate}
	\item В вакуумную камеру попадают атомы рубидия, которые испаряются на диспенсере, после чего они охлаждаются до температуры порядка нескольких сотен мкК в магнитооптической ловушке (МОЛ). Пучки МОТ формируются лазером cooling на циклическом переходе $\ket{5^2S_{1/2},F=-2,m_F=-2}\leftrightarrow \ket{5^2P_{3/2},F=-3,m_F=-3}$, дополнительно используется лазер repump для перекачки из состояния $5^2P_{1/2}$ (рис. \ref{fig:atom_scheme_raman} и \ref{fig:pulse_scheme_raman}). 
	\item Атомы из МОЛ перегружаются в массив статических оптических пинцетов, который формируется с помощью SLM лазером trap. За счёт механизмов субдоплеровского охлаждения температура атомов опускается до температуры порядка $50 \text{ мкК}$.
	\item Происходит сортировка массива атомов с помощью дополнительного подвижного оптического пинцета, который управляется с помощью АОД.
	\item Атомы инициализируются в начальное состояние $\ket{1}$ с помощью оптической накачки пучками pump и repump.
	\item Запускается последовательность квантовых логических операций. 
	\item Проводится глобальное измерение состояний кубитов по выбиванию из оптических пинцетов мощным пучком push-out.
	\item Процесс повторяется заново для накопления статистики результатов измерений.
\end{enumerate}

\vspace{2em}

\begin{center}
	\begin{tabular}{ |p{13cm}|c| } 
	  % \multicolumn{2}{|c|}{Основные этапы работы кван} \\
	  \hline
	  \textbf{Этап} & \textbf{Длительность} \\ 
	  \hline
	  Охлаждение атомов в МОЛ 							& 0.1 мс \\ 
	  Загрузка атомов в массив дипольных ловушек (массив $5 \times 5$)    & 1 с \\
	  Перемещение атома между соседними узлами массива  & 0.3 мс \\ 
	  Однокубитная логическая операция   				& 1 мкс \\
	  Двухкубитная логическая операция  				& 1 мкс \\
	  Считывание состояния 								& 50 мкс \\ 
	  \hline
	\end{tabular}
	\captionof{table}{Основные этапы работы квантового компьютера на холодных нейтральных атомах $^{87}\text{Rb}$ и их длительность.}
	\label{tab:times}
\end{center}

\begin{center}
	\begin{tabular}{ |p{13cm}| c|}
		\hline
		\textbf{Параметр} & \textbf{Значение} \\
		\hline
		Фиделити глобальной однокубитной операции с СВЧ-возбуждением (по результатам randomized benchmarking \cite{Hines_2023}) & $(99.95 \pm 0.06) \%$ \\
		\hline
		Фиделити локальной однокубитной операции с рамановским возбуждением (по скорости затухания осцилляций Раби) & $\sim 95 \%$ \\
		\hline
		Фиделити глобальной однокубитной операции с рамановским возбуждением (по результатам randomized benchmarking \cite{Hines_2023}) & $(99.56 \pm 0.03) \%$ \\
		\hline
		Фиделити глобальной двухкубитной операции CNOT до установки flat-top (по матрице истинности) & $\sim 51\%$ \\
		\hline
		Фиделити глобальной двухкубитной операции CNOT после установки flat-top (по осцилляциями чётности \cite{parity_oscillations,toffoli}) & $\sim 83\%$ \\
		\hline
		Время продольной релаксации $T_1$  &  0.5 с\\
		% \hline 
		% \hline
		Эффективное время поперечной релаксации $T_2^*$ &  2 мс\\
		\hline
	\end{tabular}
	\captionof{table}{Основные параметры квантового компьютера на холодных нейтральных атомах $^{87}\text{Rb}$ в нашей лаборатории.}
	\label{tab:params}
\end{center}
	


% 51\% - до установки флет-топов, 83 \% - после установки флет-топов



\newpage