\section{Заключение}
\label{sec:chapter_5}

\subsection{Результаты работы}

Работа состоит из двух частей одна из которых посвящена улучшению точности однокубитных логических операций, а вторая - двухкубитных. Детально результаты глав изложены в разделах \ref{sec:raman_model_results}, \ref{sec:results_chapter_4}. Основные результаты работы заключаются в следующем:

\begin{enumerate}
	\item Определен основной источник ошибок для однокубитных и двухкубитных операций с двухфотонным возбуждением осцилляций Раби, им является тепловое движение атомов в оптических пинцетах. 

	\item Сделана численная модель для рамановских двухфотонных переходов в $\Lambda$ - схеме и каскадной схеме. В модели для однокубитных операций рассматривается тепловое движение атома и спонтанный распад из промежуточного состояния. В модели для ридберговского возбуждения дополнительно рассматриваются ошибки связанные с фазовыми шумами лазеров и SPAM-ошибки. Результаты моделирования подтверждают предположение о том что основным источником ошибок является тепловое движение, а также помогают найти оптимальные экспериментальные параметры.

	\item Предложено использование последовательности BB1 для компенсации теплового движения атомов, амплитудных шумов лазера, неоднородности интенсивности по атомному массиву. Собрана схема для экспериментальной реализации на основе IQ-модулятора, начаты измерения.

	\item Предложено использование flat-top пучков для компенсации теплового движения атомов при ридберговском возбуждении, которое используется для двухкубитных логических операций. Экспериментально продемонстрировано улучшение точности возбуждения в ридберговское состояние по сравнению с использованием обычных гауссовых пучков.
\end{enumerate}

\subsection{Планы по дальнейшей работе}

В ближайшее время хотелось бы закончить экспериментальную реализацию импульсной последовательности BB1 на нашей установке, сравнить её с обычными импульсами в плане точности однокубитных вентилей. В случае улучшения качества осцилляций Раби с рамановским возбуждением хочется обобщить результаты на атомный массив, провести бенчмаркинг однокубитных операций на всём массиве. 

\newpage