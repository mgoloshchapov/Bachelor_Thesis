% \section*{Благодарности}

% \newpage

\section*{Список обозначений}
\addcontentsline{toc}{section}{Список обозначений}

\begin{tabular}{lll}
	АОМ &    & Акустооптический модулятор \\
	АОД &    & Акустооптический дефлектор \\
	МОЛ &    & Магнитооптическая ловушка \\
	SLM &    & Пространственный модулятор света \\  
	& & (Spatial Light Modulator) \\
	SPAM &    & Ошибки приготовления и измерения состояния \\
	& & (State Preparation and Measurement errors) \\
	MCMC &    & Марковские цепи Монте-Карло \\
	& & (Markov Chain Monte-Carlo) \\
	Flat-top & & Пучок с плоским распределением интенсивности в фокальной плоскости \\
	FWHM & & Полная ширина на полувысоте \\
	& & (Full Width at Half Maximum) \\
	& & \\
	& & \\
	Trap &     & Лазер на длине волны $\lambda_0 = 813 \text{ нм}$, использующийся для создания \\
	&   & массива оптических пинцетов. \\
	&   & массива оптических пинцетов. \\
	Raman seed &     & Лазер на длине волны $\lambda_R = 795 \text{ нм}$, использующийся для создания \\
	& & рамановского лазера. \\
	& & состояния атома по выбиванию. \\
	Red rydberg &    & Лазер на длине волны $\lambda_r = 795 \text{ нм}$, использующийся для возбуждения \\ 
	& & в ридберговское состояние. В главе \ref{sec:chapter_4} называется красным лазером.\\
	Blue rydberg &    & Лазер на длине волны $\lambda_r = 475 \text{ нм}$, использующися для возбуждения \\
	& & в ридберговское состояние. В главе \ref{sec:chapter_4} называется синим лазером.\\
	Push-out &     & Лазер на длине волны $780 \text{ нм}$, использующийся для детектирования \\
	& & состояния атома по выбиванию. \\
	Repump &     & Лазер на длине волны $795 \text{ нм}$, использующийся для перекачки при \\ 
	& & охлаждении в магнитооптической ловушке. \\
	& & состояния атома по выбиванию. \\
	Cooling &     & Лазер на длине волны $780 \text{ нм}$, использующийся для охлаждения в \\
	& & магнитооптической ловушке. \\
	& & состояния атома по выбиванию. \\
	Pump &     & Лазер на длине волны $795 \text{ нм}$, использующийся для оптической накачки \\
	& & в начальное состояние кубита. \\
	& & \\
	& & \\
	$w_0, z_0$ &    & Радиус перетяжки и длина Рэлея оптического пинцета \\
	$\omega_r, \omega_z$ &    & Радиальная и продольная колебательные частоты в оптической ловушке \\
	$U_0$ &    & Глубина потенциала оптического пинцета \\
	$m, \; T$ &   & Масса и температура атома $^{87}\text{Rb}$ \\
	$w_{R}, \; z_{R}, \; \Omega_{R}$ &    & Радиус перетяжки, длина Рэлея и частота Раби рамановского лазера \\
	$w_{r}, \; z_{r}, \; \Omega_{r}$ &    & Радиус перетяжки, длина Рэлея и частота Раби лазера red rydberg\\
	$w_{b}, \; z_{b}, \; \Omega_{b}$ &    & Радиус перетяжки, длина Рэлея и частота Раби лазера blue rydberg\\
	$F, \; 1-F$ & & Фиделити и инфиделити логической операции квантового компьютера
\end{tabular}


\newpage


\section{Введение}
\label{sec:Introduction} \index{Introduction}


\subsection{Квантовый компьютер на нейтральных атомах}

Квантовые компьютеры и симуляторы являются перспективными инструментами для решения сложных научных и индустриальных задач, которые возникают при моделировании новых материалов, экзотических состояний вещества \cite{Ebadi_2021,MBPColdAtoms, moire,google,Non_Abelian}, динамики сильно неравновесных квантовых процессов \cite{google,TimeCrystal_Monroe,KagomeIce}, квантовой химии \cite{QuantumChemistry}, комбинаторной оптимизации \cite{RydbergCombinatorial,Combinatorial} и многих других интересных вещей. В настоящий момент также активно развиваются новые типы алгоритмов для квантовых компьютеров, включающие в себя вариационные алгоритмы \cite{Variational}, адиабатические квантовые вычисления \cite{Adiabatic}, квантово-классические методы \cite{EnhancingVMC,QuantumClassical}. Наравне с этим появляются новые классические методы для решения задач многочастичной квантовой механики и других задач высокой размерности, среди которых можно выделить тензорные сети \cite{DMRG,TensorTrain,ITensor,perezgarcia2007matrix,DMRG_Intro}, квантовый Монте-Карло и вероятностные алгоритмы \cite{SSE,Kosztin_1996}, приближенные эвристические методы, различные подходы машинного обучения \cite{AlphaFold,AlphaGo,NNQS}. Также следует отметить, что параллельно с квантовыми компьютерами развиваются новые области альтернативной электроники, которые в будущем могут качественно улучшить возможности классических компьютеров. К таким направлениям относятся нанофотоника \cite{IntegratedPhotonics,NeuromorphicPhotonics}, спинтроника и магноника \cite{spintronics,spintronics_topo}, исследования электронных свойств низкоразмерных материалов \cite{graphene_transistors} и многое другое. 

Такие исследования, с одной стороны, расширяют возможности классических компьютеров, позволяют моделировать все более сложные и высокоразмерные системы, но, с другой стороны, всё дальше отодвигают область полезности квантовых компьютеров, предъявляют более высокие требования к точности логических операций, времени когерентности, количеству кубитов, их связности и масштабируемости. В связи с этим хочется отметить недавнее появление квантовых компьютеров на холодных нейтральных атомах, для которых уже были продемонстрированы двухкубитные операции с достоверностью $99.5\%$, реализованы поверхностные коды коррекции ошибок и порядка $50$ логических кубитов \cite{Bluvstein:2024aa}. Эта платформа хорошо зарекомендовала себя с точки зрения масштабируемости, возможности запутывать произвольные пары кубитов, гибкости в выборе топологии и расположения кубитов, возможности использования как в цифровом, так и в аналоговом режиме \cite{Ebadi_2021}, а также возможности включать и выключать взаимодействие между кубитами, что, например, не получается делать для ионов и сверхпроводников. В настоящий момент развиваются новые архитектуры квантовых процессоров на холодных нейтральных атомах на основе щёлочноземельных атомов с более интересной энергетической структурой \cite{ErasureCooling,Madjarov:2020aa,Ma:2023aa} и массивов атомов двух сортов \cite{TwoSpeciesArray}. Для таких подходов были продемонстрированы неразрушающие измерения состояний кубитов, новый тип охлаждения и кубита на нейтральных атомах \cite{ErasureCooling}, новый подход к работе с ошибками в квантовых операциях \cite{ErasureCooling,Ma:2023aa,TwoSpeciesArray}. 

По этим причинам платформа холодных нейтральных атомов кажется мне перспективной для реализации практически полезного квантового компьютера. В нашей лаборатории мы строим такой квантовый компьютер на атомах $^{87}\text{Rb}$, учимся управлять одиночными атомами в оптических пинцетах и их взаимодействиями. На текущий момент работа нашей лаборатории направлена на улучшение достоверности однокубитных и двухкубитных логических операций, увеличение размера атомного массива, реализацию быстрого перемещения и сортировки атомов, тестирование различных протоколов бенчмаркинга квантовых операций, написание компилятора квантовых схем. В моей бакалаврской работе мне бы хотелось остановиться на улучшении точности однокубитных и двухкубитных логических операций в нашей установке. 

\subsection{Цели работы}

Цели работы включают в себя:

\begin{itemize}
	\item Моделирование достоверности однокубитных операций с рамановским двухфотонным возбуждением, выделение основных источников ошибок.

	\item Улучшение однокубитных логических операций с рамановским возбуждением за счёт использования flat-top пучков. 

	\item Улучшение однокубитных логических операций за счёт использования последовательности импульсов BB1.

	\item Моделирование двухфотонного возбуждения в ридберговское состояние, оценка влияния различных источников ошибок.

	\item Улучшение двухкубитных логических операций за счёт использования flat-top пучков. 
\end{itemize}

\subsection{Актуальность работы}

На момент начала работы не была понятна причина, по которой происходит быстрое затухание осцилляций Раби при рамановском двухфотонном возбуждении, которое используется как для однокубитных, так и для двухкубитных логических операций. В процессе моделирования стало ясно, что одним из наиболее важных ограничивающих факторов является тепловое движение атома в оптической ловушке, после чего было предложено использование flat-top пучков и последовательности импульсов BB1. Использование flat-top пучков привело к существенному улучшению двухкубитных операций в нашей установке. Для последовательности BB1 было произведено моделирование, эксперимент находится в процессе реализации.

\subsection{Роль автора}

Моделирование и обработка данных проводились автором полностью самостоятельно. Получение экспериментальных данных проводилось совместно с коллегами по лаборатории при активном участии автора. 

\subsection{Структура глав} 

В главе \ref{sec:chapter_2} объясняется устройство основных этапов работы квантового компьютера на холодных нейтральных атомах: приготовление начального состояния кубита, выполнение логических операций, считывание состояния. Глава \ref{sec:chapter_3} содержит результаты по моделированию однокубитных логических операций с рамановским двухфотонным возбуждением, измеряются параметры модели, демонстрируется повышение достоверности однокубитных операций за счёт flat-top пучков и последовательности импульсов BB1. Глава \ref{sec:chapter_4} посвящена моделированию двухфотонного возбуждения в ридберговское состояние с учётом теплового движения атома в оптической ловушке, спонтанного распада из промежуточного состояния, фазовых шумов лазера, а также ошибок приготовления и измерения состояния. Экспериментально демонстрируется улучшение точности двухкубитных операций за счёт использования flat-top пучков. В главе \ref{sec:chapter_5} подводятся итоги работы, обсуждаются дальнейшие планы.

\newpage