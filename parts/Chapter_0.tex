% \section*{Благодарности}

% \newpage

\section*{Список обозначений}
\addcontentsline{toc}{section}{Список обозначений}

\begin{tabular}{lll}
	АОМ &    & Акустооптический модулятор \\
	АОД &    & Акустооптический дефлектор \\
	МОЛ &    & Магнитооптическая ловушка \\
	SLM &    & Пространственный модулятор света \\  
	& & (Spatial Light Modulator) \\
	SPAM &    & Ошибки приготовления и измерения состояния \\
	& & (State Preparation and Measurement errors) \\
	MCMC &    & Марковские цепи Монте-Карло \\
	& & (Markov Chain Monte-Carlo) \\
	Flat-top & & Пучок с плоским распределением интенсивности в фокальной плоскости \\
	& & \\
	& & \\
	Trap &     & Лазер на длине волны $\lambda_0 = 813 \text{ нм}$, использующийся для создания \\
	&   & массива оптических пинцетов. \\
	Red rydberg &    & Лазер на длине волны $\lambda_r = 795 \text{ нм}$, использующийся для возбуждения \\ 
	& & в ридберговское состояние. В главе \ref{sec:chapter_4} называется красным лазером.\\
	Blue rydberg &    & Лазер на длине волны $\lambda_r = 475 \text{ нм}$, использующися для возбуждения \\
	& & в ридберговское состояние. В главе \ref{sec:chapter_4} называется синим лазером.\\
	& & \\
	& & \\
	$w_0, z_0$ &    & Радиус перетяжки и длина Рэлея оптического пинцета \\
	$\omega_r, \omega_z$ &    & Радиальная и продольная колебательные частоты в оптической ловушке \\
	$U_0$ &    & Глубина потенциала оптического пинцета \\
	$T$ &    & Температура атома \\
	$w_{R}, z_{R}$ &    & Радиус перетяжки и длина Рэлея рамановского лазера \\
	$m$ &    & Масса атома $^{87}\text{Rb}$ \\
	$w_r, \; w_b$ &    & Радиусы перетяжки красного и синего лазеров
\end{tabular}




\newpage



\section{Введение}
\label{sec:Introduction} \index{Introduction}


\subsection{Квантовый компьютер}



\subsection{Цели работы}

Цели работы, как это часто бывает, сформировались во время работы. Основные задачи автора включали в себя:

\begin{itemize}
	\item Моделирование достоверности однокубитных операций с рамановским двухфотонным возбуждением, выделение основных источников ошибок.

	\item Улучшение однокубитных логических операций с рамановским возбуждением за счёт использования flat-top пучков. 

	\item Улучшение однокубитных логических операций за счёт использования последовательности импульсов BB1.

	\item Моделирование двухфотонного возбуждения в ридберговское состояние, оценка влияния различных источников ошибок.

	\item Улучшение двухкубитных логических операций за счёт использования flat-top пучков. 
\end{itemize}

\subsection{Актуальность работы}

На момент начала работы не была понятна причина, по которой происходит быстрое затухание осцилляций Раби при рамановском двухфотонном возбуждении, которое используется как для однокубитных, так и для двухкубитных логических операций. В процессе моделирования стало ясно, что одним из наиболее важных ограничивающих факторов является тепловое движение атома в оптической ловушке, после чего было предложено использование flat-top пучков и последовательности импульсов BB1. Использование flat-top пучков привело к существенному улучшению двухкубитных операций в нашей установке. Для последовательности BB1 было произведено моделирование, эксперимент находится в процессе.

\subsection{Роль автора}

Моделирование и обработка данных проводились автором полностью самостоятельно. Получение экспериментальных данных проводилось совместно с коллегами по лаборатории при активном участии автора. Резюмируя, у автора была роль, и он её придерживался.

\subsection{Структура глав} 

В главе \ref{sec:chapter_2} объясняется устройство основных этапов работы квантового компьютера на холодных нейтральных атомах: приготовление начального состояния кубита, выполнение логических операций, считывание состояния. Глава \ref{sec:chapter_3} содержит результаты по моделированию однокубитных логических операций с рамановским двухфотонным возбуждением, измеряются параметры модели, демонстрируется повышение достоверности однокубитных операций за счёт flat-top пучков и последовательности импульсов BB1. Глава \ref{sec:chapter_4} посвящена моделированию двухфотонного возбуждения в ридберговское состояние с учётом теплового движения атома в оптической ловушке, спонтанного распада из промежуточного состояния, фазовых шумов лазера, а также ошибок приготовления и измерения состояния. Экспериментально демонстрируется улучшение точности двухкубитных операций за счёт использования flat-top пучков. В главе \ref{sec:chapter_5} подводятся итоги работы, обсуждаются дальнейшие планы.

\newpage